\documentclass[12pt,letterpaper]{book}
\usepackage[utf8x]{inputenc}
\usepackage[spanish]{babel}
\usepackage{latexsym,amsmath,amssymb,amsthm}
\usepackage{graphicx}
\usepackage{pifont}
\usepackage[pdftex=true,colorlinks=true,plainpages=false]{hyperref}
\hypersetup{urlcolor=blue}
\hypersetup{linkcolor=black}
\hypersetup{citecolor=black}
\usepackage{lastpage}
\usepackage{url}
\usepackage{anysize} 
\marginsize{2.5cm}{1.5cm}{1cm}{1.2cm}
\usepackage{fancyhdr}
\usepackage{anyfontsize}
\usepackage{tocbibind}
\usepackage{multicol}
%%%% “”\slshape 

\pagestyle{fancy}
\lhead{\small \slshape \leftmark} \chead{} \rhead{\small \slshape \rightmark}
\lfoot{\url{http://scesi.fcyt.umss.edu.bo}} \cfoot{\includegraphics[scale=0.25]{gnome/logo.png}} \rfoot{\thepage/\pageref{LastPage}}
\renewcommand{\headrulewidth}{0.4pt}
\renewcommand{\footrulewidth}{0.3pt}
\setlength{\headsep}{0.9\headsep}
\setlength{\footskip}{1.3\footskip}

%\title{Gnome}
%\author{Ubaldino Zurita}
%\date{\today}

\begin{document}
 \begin{titlepage}
	\thispagestyle{empty}
	\begin{center}
		\includegraphics[scale=0.7]{gnome/gnome-3.jpg} \\
		~\\
		\Large{\textsc{\bf Universidad Mayor De San Simón}}\\
		\large{\textsc{\bf Facultad De Ciencias y Tecnológica}}\\
		\large{\textsc{\bf Sistemas e Informática}}\\
		~\\
		\small{\bf \today}
	\end{center}
 	\vfill
	\begin{center}
		\Huge{\textsc{\bf Gnome}}
	\end{center}
	\vfill
	\hrule
	\vspace{0.1cm}
	\noindent\small{\url{http://scesi.fcyt.umss.edu.bo} \hfill \url{http://www.memi.umss.edu.bo}}
	\hrule
	\vspace{0.1cm}
	\noindent\small{\hspace{1.15cm}\includegraphics[scale=0.06]{gnome/scesi.png} \hfill \includegraphics[scale=0.23]{gnome/memi.jpg}\hspace{0.83cm}}

\end{titlepage}

%\maketitle
\tableofcontents
%\setcounter{page}{1}
\part{INICIO DE SESIÓN}
\chapter{Pantalla de inicio de sesión}
\section{La pantalla de inicio de sesión incluye los siguientes elementos}
\begin{description}
\item[Solicitud de entrada] Escriba su nombre de usuario y contraseña para iniciar la sesión.
\item[Menú Idioma] Seleccione un idioma para la sesión.
\item [Menú Tipo de sesión] Seleccione el escritorio que se debe ejecutar durante la sesión. Si hay instalados otros escritorios, aparece
\item[Menú Idioma] Seleccione un idioma para la sesión.
\item [Menú Tipo de sesión] Seleccione el escritorio que se debe ejecutar durante la sesión. Si hay instalados otros escritorios, aparecerán en la lista.\\ pero en nuestro caso es {\bf Gnome 3.}
\item[Reiniciar] Reinicia el equipo.
\item[Apagar] Apaga el equipo.
\end{description}
\subsubsection{Inicio de sesión}
Una sesión es la entrada de un usuario a su cuenta en el sistema operativo instalado la cual tiene una pantalla de entrada que ofrece varias opciones relacionadas.\\
Por ejemplo, se puede seleccionar el idioma de la sesión, de modo que el texto de la interfaz aparezca en el idioma seleccionado.\\
Una vez que se autentican el nombre de usuario y la contraseña, se inicia el administrador de sesiones, el cual permite guardar determinados valores de configuración de cada sesión. También permite guardar el estado de la sesión más reciente y recuperarla la próxima vez que entre.\\
El administrador de sesiones permite guardar y restaurar los siguientes ajustes:
\begin{itemize}
\item[$\bullet$] Configuración de visualización y funcionamiento del escritorio, como fuentes, colores y configuración del ratón.
\item[$\bullet$] Aplicaciones que se están ejecutando, como un gestor de archivos o un programa de OpenOffice.org.
\end{itemize}
{\bf \underline{A tener en cuenta}}\\
No es posible guardar ni restaurar aplicaciones que no gestione el administrador de sesiones. Por ejemplo, si se inicia el editor vim en la línea de comandos de una ventana de terminal, el administrador de sesiones no puede restaurar la sesión de edición.

\section{Salida de sesión}
Cuando haya terminado de usar el equipo, puede salir de la sesión y dejar el sistema en ejecución, o bien reiniciar o apagar el equipo.\\
\subsection{Salida de la sesión o cambio de usuario}
\begin{enumerate}
\item Haga clic en Ordenador $=>$ Salir.
\item Seleccione una de las siguientes opciones:
\begin{description}
\item[Salir de la sesión] Se sale de la sesión en curso y se vuelve a la pantalla de entrada a la sesión.
\item[Cambiar de usuario] Se suspende la sesión y se permite que otro usuario entre y utilice el equipo.
\end{description}
\end{enumerate}
\subsection{Reinicio o cierre del equipo}
\begin{enumerate}
\item Haga clic en Ordenador $=>$ Apagar. 
\item Seleccione una de las siguientes opciones:
\begin{description}
\item[Apagar] Se cierra la sesión actual y se apaga el equipo.
\item[Reiniciar] Se cierra la sesión actual y se reinicia el equipo.
\item[Reposo] El equipo adopta un estado temporal de ahorro de energía. Se mantiene el estado de la sesión, incluidas todas las aplicaciones en ejecución y todos los documentos abiertos.
\item[Hibernar] Se suspende la sesión y no se emplea electricidad hasta que se reinicia el equipo. Se mantiene el estado de la sesión, incluidas todas las aplicaciones en ejecución y todos los documentos abiertos.
\end{description}
\end{enumerate}
%------------------------------------------------------
\part{GNOME}
\chapter{Contenido y Uso de Gnome Shell}
\begin{center}
\includegraphics[scale=0.448]{gnome/barra.png}
\end{center}
Gnome es uno de los tantos escritorios que existen para las distribuciones basadas en linux.
En este caso UBUNTU 11.10 utiliza el escritorio o interfaz gráfica Gnome 3, y como es de esperar de un escritorio, los componentes principales del escritorio Gnome son:
\begin{itemize}
\item[-] Iconos que enlazan a archivos, carpetas o programas (accesos directos).
\item[-] Gnome Shell que es la barra de tareas y lanzador de aplicaciones. y tiene un gestor de ventanas con soporte OpenGL.
\begin{description}
\item[Gnome Shell] tiene un panel único en la parte superior, que contiene:
\begin{itemize}
 \item menú de usuario. 
 \item un botón “actividades”. 
 \item aplicaciones en ejecución.
 \item reloj.
 \item área de accesibilidad.
 \item control de volumen.
 \item información de la red.
 \item estado de la batería en el caso de un portátil.
\end{itemize}
En el arranque del sistema aparece dicho panel y el fondo de escritorio.
\end{description}
\includegraphics[scale=0.35]{gnome/Pantallazo3.png}\\ 
Al pulsar sobre “actividades” o acercar el ratón con rapidez a la esquina superior izquierda, se despliegan varios elementos nuevos:
\begin{itemize}
\item A la izquierda, un panel lanzador de aplicaciones,
\item Bajo el panel superior, un menú de dos opciones, “ventanas” y “aplicaciones”.
\item A la derecha, otro panel con una vista previa de las ventanas activas que actúa como selector de escritorios y por encima de éste, una caja de búsqueda.
\end{itemize}
Si tienes ya algún programa en ejecución, que por defecto se lanza a pantalla completa, la ventana o ventanas activas que su reducen su tamaño mostrando una leyenda informativa en la parte inferior de cada ventana. Cada acción que comporte cambios de tamaño o posición, vendrá acompañada de una elegante animación y otros efectos visuales.\\
El botón gráfico windows (ventanas), restaura la vista previa de aquellas que estén activas si las has perdido de vista por ejecutar cualquier otra acción.\\
\includegraphics[scale=0.35]{gnome/Pantallazo4.png}\\ 

El botón aplications (aplicaciones), da paso a la representación mediante iconos de los programas instalados en la máquina.\\

\includegraphics[scale=0.35]{gnome/Pantallazo5.png} \\
\underline{\bf Nota:} Note que cuando estamos en la presentación de los programas instalados en la maquina. El panel de la derecha porta un sistema de filtrado, clasificando los programas según las funcionalidades.\\

La caja de búsqueda es lo novedoso de gnome 3. Escribiendo las primeras letras de una aplicación, aparecerá de forma inmediata el icono correspondiente al programa y puedes lanzar la aplicación desde ahí. Y en la parte inferior de la pantalla muestra dos botones gráficos, que permiten extender la búsqueda en Wikipedia y Google.\\ 

\includegraphics[scale=0.35]{gnome/Pantallazo6.png}\\ 

Por último, terminando nuestro recorrido en Gnome Shell, en la parte inferior derecha de la pantalla veremos el área de notificaciones y bandeja de mensajes, desde donde puedes contestar con el programa de mensajería instantánea.\\
\includegraphics[scale=0.35]{gnome/Pantallazo7.png} 
\end{itemize}
\chapter{barra superior}
\begin{center}
\includegraphics[scale=0.45]{img/barra2.png} 
\end{center}
GNOME 3 cuenta con una interfaz de usuario completamente reinventada, diseñada para permanecer fuera de su vista, minimizar las distracciones, y ayudarle a trabajar. La primera vez que inicie una sesión, verá un escritorio vacío y la barra superior.\\

La barra superior permite el acceso a las ventanas y aplicaciones, su calendario y citas, y las propiedades del sistema como el sonido, la red, y la energía. Bajo su nombre en la barra superior, puede establecer su disponibilidad, cambiar su perfil o configuración, salir o cambiar de usuario, o apagar el equipo.
\section{Vista de actividades}
Para acceder a sus ventanas y aplicaciones, pulse el botón Actividades, o simplemente lleve el puntero del ratón a la esquina activa. También puede pulsar la tecla Windows en su teclado. Puede ver sus ventanas y aplicaciones en la vista. También puede empezar a escribir para buscar aplicaciones, archivos o carpetas.\\

A la izquierda de la vista, encontrará el tablero. El tablero le muestra sus aplicaciones favoritas y en ejecución. Pulse en cualquier icono en el tablero para abrir dicha aplicación. Si la aplicación se está ejecutando, pulsar en el icono abrirá la ventana utilizada ​​más recientemente.\\
Para seleccionar una ventana en una aplicación en ejecución, o para abrir una ventana nueva, pulse con el botón derecho del ratón sobre la aplicación. También puede arrastrar el icono sobre la vista general, o sobre cualquier miniatura de área de trabajo a la derecha.
Cuando entre en la vista, inicialmente estará en la vista de las ventanas. Esto muestra las miniaturas de todas las ventanas en el área de trabajo actual. Pulse en cualquier ventana para dar el foco la ventana y salir de la vista. También puede usar la rueda de desplazamiento del ratón para aumentar cualquier miniatura de la ventana.\\
Pulse en Aplicaciones para entrar en la vista de aplicaciones. Esto le muestra todas las aplicaciones instaladas en su equipo. Pulse en cualquier aplicación para ejecutarla, o arrastre una aplicación a la vista o sobre la miniatura del espacio de trabajo. También puede arrastrar una aplicación al tablero para que sea uno de los favoritos. Sus aplicaciones favoritas permanecerán en el tablero, incluso cuando no estén en funcionamiento, para que pueda acceder a ellas rápidamente.
\section{Reloj, calendario y citas}
\begin{center}
\includegraphics[scale=0.5]{img/calendario.png} 
\end{center}
Pulse en el reloj en el centro de la barra superior para ver la fecha actual, un calendario mensual y una lista de sus próximas citas. También puede acceder a la configuración de fecha y hora y abrir totalmente su calendario de Evolution directamente desde el menú.
\subsection{Citas de calendario}
Esto requiere que Evolution esté instalado en su equipo.\\
La mayoría de las distribuciones tienen instalado Evolution de forma predeterminada. Si la suya no lo tiene, puede querer instalarlo usando el gestor de paquetes de su distribución.\\

{\bf Para ver sus citas:}
\begin{enumerate}
\item Pulse en el reloj en la barra superior.
\item Seleccione en el Calendario la fecha para la que quiere ver sus citas.
\item El calendario mostrará las citas existentes a la derecha. Ya que las citas se añaden en Evolution, éstas aparecerán en la lista de citas del reloj.\\
%%%imagen

Para conseguir rápidamente le calendario completo de Evolution, pulse en el reloj y pulse Abrir calendario.\\
Esto funcionará solo si tiene una cuenta de Evolution. En caso contrario, aparecerá una ventana con los pasos necesarios para añadir su primera cuenta.
\end{enumerate}
\section{Usted y su equipo}
\begin{center}
\includegraphics[scale=0.45]{img/usuario.png} 
\end{center}
Pulse en su nombre de usuario en la esquina superior derecha de la pantalla para gestionar su perfil y su equipo.\\
Puede establecer rápidamente su disponibilidad directamente desde el menú. Cuando usa la aplicación de mensajería instantánea Empathy, esto establecerá su estado para que sus contactos lo vean.\\

El menú también le permite editar su información personal y cambiar la configuración del sistema.\\

Al dejar su equipo, puede bloquear la pantalla para evitar que otras personas lo usen. Puede rápidamente cambiar de usuario sin necesidad de iniciar la sesión completamente para dar a alguien acceso al equipo. O bien, puede suspender o apagar el equipo desde el menú.
%%%%%imagen
\subsection{Cerrar la sesión, apagar o cambiar de usuario}
Cuando haya terminado de usar su equipo, puede apagarlo, suspenderlo (para ahorrar energía), o dejarlo encendido y cerrar la sesión.
\subsubsection{Cerrar la sesión o cambiar de usuario}
Para permitir que otros usuarios usen su equipo, puede salir de la sesión, o seguir conectado y simplemente cambiar de usuario. Si cambia de usuario, todas las aplicaciones seguirán funcionando y todo estará donde lo dejó cuando vuelva a iniciar sesión.\\
Para Salir de la sesión o Cambiar de usuario, pulse sobre su nombre en la barra superior y seleccione la opción apropiada.
\subsubsection{Bloquear la pantalla}
Si deja su equipo durante un breve periodo de tiempo, debe bloquear la pantalla para evitar que otras personas tengan acceso a sus archivos y ejecuten aplicaciones. Cuando vuelva, simplemente introduzca su contraseña para volver a iniciar la sesión. Si no bloquea la pantalla, se bloqueará automáticamente tras un cierto tiempo.\\

Para bloquear la pantalla, pulse sobre su nombre en la barra superior y seleccione Bloquear la pantalla.\\

Cuando su pantalla está bloqueada otros usuarios pueden iniciar sesión en sus propias cuentas pulsando Cambiar de usuario en la pantalla de contraseña. Puede volver a su escritorio cuando hayan terminado.
\subsubsection{Suspender}
Para ahorrar energía, suspenda su equipo cuando no lo esté usando. Si usa un equipo portátil, GNOME suspende su equipo automáticamente cuando cierra su tapa. Esto guarda su estado en la memoria de su equipo y apaga la mayor parte de las funciones de su equipo. Durante la suspensión se sigue usando una cantidad muy pequeña de energía.\\

Para suspender su equipo manualmente, pulse sobre su nombre en la barra superior y seleccione Suspender.
\subsubsection{Apagar o reiniciar}
Si quiere apagar su equipo por completo o hacer un reinicio total, primero cierre la sesión pulsando en su nombre en la barra superior y seleccionando Cerrar sesión. Volverá a la pantalla de inicio de sesión. En esta pantalla, pulse el icono de encendido en la barra superior y seleccione Reiniciar o Apagar.\\
Si hay otros usuarios conectados, no podrá apagar o reiniciar el equipo, porque esto cerraría sus sesiones. Si es un usuario administrativo, se le pedirá su contraseña para apagar.\\

Si necesita apagar o reiniciar rápidamente lo puede hacer sin cerrar la sesión. Pulse en su nombre, en la barra superior y mantenga pulsada la tecla Alt. La opción Suspender cambiará a Apagar. Seleccione apagar o reiniciar.
\section{Bandeja de mensajes}
\begin{center}
\includegraphics[scale=0.5]{img/bandeja.png} 
\end{center}
La bandeja de mensajes puede mostrarse moviendo el ratón a la esquina inferior derecha. Aquí es donde se almacenan las notificaciones hasta que esté listo para verlas.
\subsection{¿Qué es una notificación?}
Si una aplicación o un componente del sistema quiere llamar su atención, mostrará una notificación en la parte inferior de la pantalla.\\
Por ejemplo, si recibe un nuevo mensaje de chat, hay nuevas actualizaciones disponibles para su equipo o la batería está baja, recibirá una notificación informándole de ello.\\

La notificación aparecerá primero como una sola línea, de modo que no le distraiga. Puede mover el ratón sobre ella si quiere ver su contenido completo.

\subsection{Ocultar notificaciones}
Si está trabajando en algo y no quiere que le molesten, puede desactivar las notificaciones. Simplemente pulse en su nombre en la barra superior y seleccione Notificaciones para cambiarlo a Apagado.\\
Cuando esté apagado, la mayoría de las notificaciones no se mostrarán como mensajes emergentes en la parte inferior de la pantalla. Las notificaciones importantes, tales como batería críticamente baja, se seguirán mostrando. Las notificaciones seguirán estando disponibles en la bandeja de mensajería al mover el ratón a la esquina inferior derecha, y se mostrarán cuando seleccione estar disponible de nuevo.
\chapter{Uso de application launcher}
\begin{center}
\includegraphics[scale=0.35]{gnome/Pantallazo9.png} 
\end{center}
En este capitulo aprenderemos la forma mas rápida de iniciar aplicaciones\\

Mueva el puntero de su ratón a la esquina de Actividades en la parte superior izquierda de la pantalla para mostrar la Vista de actividades. Aquí es donde puede encontrar todas sus aplicaciones. (También puede abrir la vista pulsando la supertecla o tecla windows.)
Hay distintas maneras de abrir una aplicación una vez que está en la vista de actividades:
\begin{itemize}
\item Comience a escribir las primeras letras de una aplicación; y la búsqueda comenzará al instante. (Si esto no sucede, pulse en la barra de búsqueda en la parte superior derecha de la pantalla y comience a escribir.) Pulse en el icono de la aplicación para iniciarla.
\item Pulse en la cabecera de Aplicaciones en la parte superior de la pantalla para ver una lista de aplicaciones que puede ejecutar. Puede filtrar por tipo, usando las categorías de la derecha, o buscar mediante la barra de búsqueda en la parte superior derecha. Pulse en el icono de la aplicación para iniciarla.
\item Algunas aplicaciones tienen iconos en el tablero, la franja vertical de los iconos en el lado izquierdo de la vista de actividades. Pulse en uno de ellos para iniciar la aplicación correspondiente.\\
Si tiene aplicaciones que usa muy frecuentemente, puede añadirlas al tablero.
\item Puede iniciar una aplicación en un área de trabajo independiente arrastrando el icono de la aplicación desde el tablero (o desde la lista de aplicaciones), y colocándolo en una de las áreas de trabajo en la parte derecha de la pantalla. La aplicación se abrirá en el área de trabajo que elija.
\end{itemize}
Por ejemplo, para lanzar el gestor de archivos, escriba “nautilus” (sin las comillas). El nombre de la aplicación es el comando para lanzar el programa.

\chapter{Acceso a carpetas}
\includegraphics[scale=0.6]{gnome-3-nautilus.jpg}
Existen varios gestores de archivos en linux.
Nautilus, es el gestor gráfico de archivos de Gnome, muy fácil de usar.
\begin{itemize}
\item Tiene barra lateral, que se ve muy bien y permite organizar tus “lugares” de forma más sencilla y fácil para navegar por el árbol de directorios.
\item Incorpora una barra de herramientas novedosa e información inteligente de estado.
\item Puede conectarse a un servidor de archivos.
\end{itemize}
A continuación veremos las diferentes tareas de nautilus:
\section{Tareas comunes}
\subsubsection{Examinar archivos y carpetas}
Use la aplicación Archivos para navegar y organizar los archivos en su equipo. También puede usarlo para gestionar archivos en dispositivos de almacenamiento (como discos externos), en servidores de archivos y en recursos compartidos de la red.\\
Para abrir el gestor de archivos, entras en actividades. escribes nautilus y enter si te sale el gestor de archivos.\\
Para pulse acceder a cualquier carpeta pulsa dos veces sobre una carpeta para ver su contenido, o sobre un archivo para abrirlo con la aplicación predeterminada para ese archivo. También puede pulsar con el botón derecho sobre una carpeta para abrirla en una pestaña o una ventana nueva.\\ 
En la vista de lista, también puede pulsar en el extensor situado junto a la carpeta para mostrar su contenido en forma de árbol.\\
Al examinar los archivos en una carpeta puede previsualizar rápidamente cada archivo rápidamente pulsando la barra espaciadora para asegurarse de que tiene el archivo correcto antes de abrirlo, copiarlo o eliminarlo.\\
La barra de rutas situada encima de la lista de archivos y carpetas muestra qué carpeta esta visualizando, incluyendo sus carpetas padre hasta su carpeta personal. Pulse en una carpeta padre de la barra de rutas para moverse a esa carpeta. Pulse con el botón derecho sobre cualquier carpeta de la barra de rutas para abrirla en una pestaña o ventana nueva, copiarla, moverla o acceder a sus propiedades.\\
Si quieres saltar rápidamente a un archivo en la carpeta que está viendo, empiece a escribir su nombre. Aparecerá una caja de búsqueda en la parte inferior de la ventana y se resaltará el primer archivo que coincida con su búsqueda. Pulse la flecha abajo , Ctrl+G o desplácese con el ratón para saltar al siguiente archivo que coincida con su búsqueda.\\
Puede acceder rápidamente a lugares comunes desde la barra lateral. Si no ve la barra lateral, pulse en Ver $>$ Barra lateral $>$ Mostrar barra lateral. Puede añadir marcadores a carpetas que use con frecuencia y aparecerán en la barra lateral. Use el menú Marcadores para hacer esto, o simplemente arrastre una carpeta a la barra lateral.\\
Si mueve frecuentemente archivos entre carpetas anidadas, puede encontrar útil mostrar un árbol en la barra lateral. Pulse Ver $>$ Barra lateral $>$ Árbol, para activar el árbol de la barra lateral. Pulse el expansor junto a la carpeta para mostrar sus carpetas hijas en el árbol, o pulse en una carpeta para abrirla en la ventana.
\subsubsection{Buscar archivos}
Puede buscar archivos según su nombre o su tipo de archivo directamente desde el gestor de archivos. También puede guardar búsquedas frecuentes, que aparecerán como carpetas especiales dentro de su carpeta personal.\\

{\large \bf Buscar}
\begin{enumerate}
\item Abra la aplicación Archivos desde la vista de Actividades.
\item Si sabe que los archivos que quiere buscar están en una carpeta determinada, vaya a esa carpeta.
\item Pulse Buscar en la barra de herramientas, o pulse Ctrl+F.
\item Teclee una o varias palabras que sepa que aparecen en el nombre del archivo y pulse Intro. Por ejemplo, si todas sus facturas contienen en su nombre la palabra Factura, teclee factura. Pulse Intro No hace falta tener en cuenta las mayúsculas y minúsculas.
\item Puede acotar los resultados por ubicación y tipo de archivo. Pulse el botón + para establecer más criterios de búsqueda.
\begin{itemize}
\item Para acotar los resultados de la búsqueda, seleccione Ubicación de la lista desplegaba para añadir una ubicación padre.
\item Para acotar los resultados de la búsqueda en función del tipo de archivo, seleccione Tipo de archivo de la lista desplegable.
Pulse el botón - junto a cualquier opción de búsqueda para quitar esa opción y aumentar los resultados de búsqueda.
\end{itemize}
\item Puede abrir, copiar, eliminar o trabajar con sus archivos desde los resultados de búsqueda, igual como si estuviera en cualquier carpeta en el gestor de archivos.
\item Pulse de nuevo Buscar en la barra de herramientas para salir de la búsqueda y volver a la carpeta.
\end{enumerate}
Si lleva a cabo ciertas búsquedas muy a menudo, puede guardarlas para acceder a ellas rápidamente.\\

{\large \bf Guardar una búsqueda}
\begin{enumerate}
\item Iniciar una búsqueda como la de arriba.
\item Cuando esté satisfecho con sus parámetros de búsqueda, pulse Archivo $>$ Guardar búsqueda como.
\item Asigne un nombre a la búsqueda y pulse Guardar. Si lo desea, seleccione una carpeta distinta en la que guardarla. Cuando visualice esa carpeta, verá sus búsquedas guardadas como un icono de carpeta color naranja con una lupa.
\end{enumerate}
Para eliminar el archivo buscado cuando haya terminado con él, simplemente elimínelo de la búsqueda igual que haría con cualquier otro archivo. Cuando elimina una búsqueda guardada, no elimina los archivos que coincidieron con la búsqueda.

\subsubsection{Copiar o mover archivos y carpetas}
\subsubsection{Eliminar archivos y carpetas}
\subsubsection{Ordenar archivos y carpetas}
\subsubsection{Previsualizar archivos y carpetas}
\subsubsection{Renombrar un archivo o una carpeta}
\section{Otros temas}
\subsubsection{Abrir archivos con otras aplicaciones}
\subsubsection{Compartir y transferir archivos}
\subsubsection{Encontrar un archivo perdido}
\subsubsection{Escribir archivo en un CD o DVD}
\subsubsection{Examinar archivos en un servidor o comparticn de red}%%%%%tilde
\subsubsection{Propiedades del archivo}
\subsubsection{Recuperar un archivo eliminado}
\subsubsection{Preferencias del gestor de archivos}


\input{personalizacion}
\end{document}
